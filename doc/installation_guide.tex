%% -*- mode:latex; tex-open-quote:"\\og{}"; tex-close-quote:"\\fg{}" -*-
%%
%%  Copyright (c) 2002-2006 by Pascal Vincent and Olivier Delalleau
%%
%%  $Id: users_guide.tex 6000 2006-07-11 18:36:50Z plearner $

\documentclass[11pt]{book}
% \usepackage[latin1]{inputenc}
\usepackage{times}
\usepackage{html}               % package for latex2html
% \usepackage{t1enc}              % new font encoding  (hyphenate words w/accents)
% \usepackage{ae}                 % use virtual fonts for getting good PDF
\usepackage{hyperref}
\usepackage{verbatim}

%%%%%%%%%%%%%%%%%%%%%%%%%%%%%%%%%%%%%%%%%%%%%%%%%%%%%%%%%%%%%%%%%%%%%%%%%%%%%%%
%%%  Support for PDF and required packages
%%%%%%%%%%%%%%%%%%%%%%%%%%%%%%%%%%%%%%%%%%%%%%%%%%%%%%%%%%%%%%%%%%%%%%%%%%%%%%%

\usepackage{graphicx}	% enhanced pictures import (w/ PDF)


%%% %%% Basic switch to detect whether to use PDF or not
%%% \newif\ifpdf
%%% \ifx\pdfoutput\undefined
%%%   \pdffalse
%%%   \newcommand{\mypdfoption}{{}}
%%% \else
%%%   \pdfoutput=1
%%%   \pdftrue
%%%   \newcommand{\mypdfoption}{pdftex}
%%%   \pdfcompresslevel=9
%%% \fi
%%% 
%%% %%% Packages that behave differently in PDF and non-PDF
%%% \usepackage[\mypdfoption]{graphicx}	% enhanced pictures import (w/ PDF)
%%% \usepackage[\mypdfoption]{rotating}	% rotate stuff on the page
%%% \usepackage[\mypdfoption]{color}        % color support
%%% \usepackage[\mypdfoption]{colortbl}     % Add color and shading to tables
%%% 
%%% %%% Declare graphics file extensions as .pdf or .eps for \includegraphics.
%%% %%% (You must NOT specify any extension in the filenames!  \includegraphics
%%% %%% will by itself determine the correct extension.)
%%% \ifpdf
%%%   \DeclareGraphicsExtensions{.jpg,.png,.pdf,.mps} % .png before .pdf ==> okay!
%%% \else
%%%   \DeclareGraphicsExtensions{.jpg,.eps,.png,.mps} % .eps before .png ==> okay!
%%% \fi


%%%%%%%%% Definitions %%%%%%%%%%%%
\newcommand{\R}{\sf{I\!R}} 

\newcommand{\PLearn}{{\bf \it PLearn}~}
\newcommand{\Object}{{\bf Object}~} 
\newcommand{\Learner}{{\bf Learner}~} 
\newcommand{\PPointable}{{\bf PPointable}~} 
\newcommand{\VMatrix}{{\bf VMatrix}~} 
\newcommand{\VMat}{{\bf VMat}~} 
\newcommand{\PLearner}{{\bf PLearner}~} 

\parskip=2mm
\parindent=0mm

\title{{\Huge PLearn Installation Guide\\ \Large How to install the PLearn Machine-Learning library and tools}}

\begin{document}

%%%%%%%% Title Page %%%%%%%%%%
\pagenumbering{roman}

\maketitle

\vspace*{8cm}

Copyright \copyright\ 1998-2006 Pascal Vincent \\
Copyright \copyright\ 2005-2006 Olivier Delalleau \\

Permission is granted to copy and distribute this document in any medium,
with or without modification, provided that the following conditions are
met:

\begin{enumerate}
\item Modified versions must give fair credit to all authors.
\item Modified versions may not be written with the aim to discredit, misrepresent, or otherwise taint the
      reputation of any of the above authors.
\item Modified versions must retain the above copyright notice, and append to
   it the names of the authors of the modifications, together with the years the
   modifications were written.
\item Modified versions must retain this list of conditions unaltered, 
    and may not impose any further restrictions.
\end{enumerate}

%%%%%%%%% Table of contents %%%%%%%%%%%%
\addcontentsline{toc}{chapter}{\numberline{}Table of contents}
\tableofcontents

% \begin{latexonly}
% \cleardoublepage\pagebreak
% \end{latexonly}
% \pagenumbering{arabic}

\pagenumbering{arabic}

%%%%%%%%%%%%%%%%%%%%%%%%%%%%%%%%%%%%%%%%%%%%%%%%%%%%%%%%%%%%%%%%%%%%%%%%%%%%%%%

\chapter{Overview of requirements}
\label{chap:label}

{\bf Note:} most of the tools and libraries required for PLearn are already
installed on typical Linux (or other unix-like) systems, or are easy to
install with ready-made packages (such as RPMs or using apt-get). PLearn is mostly
developped in a Linux environment with the gcc/g++ C++ compiler.  Since
PLearn was written with portability in mind, it is certainly
portable (and has been ported on occasions) to other platforms or
compilers, but as we work mainly with Linux/gcc, we only very occasionally
check the working and correct problems on other platforms.


\section{Required tools}

To be able to download, compile and use PLearn, you need the following
tools to be installed on your system (for detailed instructions for
installation under Windows, you may go directly to section~\ref{sec:windows}).
\begin{itemize}
\item {\bf g++} (\url{http://gcc.gnu.org}). We recommend using the latest version,
  but it should work with 3.2 and above. It is certainly possible to
  compile PLearn with other compilers (we do this from time to time), but
  it may necessitate some tweaking.
\item {\bf python} (\url{http://www.python.org}). Version 2.3.3 or above (may
  work with older versions, but no guarantee). We use python scripts
  heavily to make our life easier, noticeably as our compilation framework
  (but if you're really {\em in love} with Makefiles you can probably do without
  pymake).
\item {\bf subversion (svn)} for version control ( see \url{http://subversion.tigris.org/} )
Note: some local subprojects may still be based on CVS for version control (\url{http://www.cvshome.org})
\end{itemize}


\section{External library dependencies}

Parts of PLearn depend on the following external libraries. Depending on
which parts of PLearn you want to compile and use, they may require
compiling and linking with the following libraries (that will first have to
be installed: see more detailed installation instructions for your system
in the appropriate chapter).

\begin{itemize}
\item {\bf required:} the standard C and C++ libraries, naturally!
\item {\bf required:} {\bf NSPR} the Netscape Portable Runtime
  (http://www.mozilla.org/projects/nspr/). We use NSPR as an OS abstraction
  layer for things such as file and network access, process control,
  etc... So that PLearn may also somewhat work on Windows...
\item {\bf required:} parts of PLearn have come to depend on some of the
  {\bf boost} C++ libraries (http://www.boost.org). You might as well install all
  of boost. We use in particular: {\em tuple, random, type\_traits, call\_traits, smart\_ptr, bind, function, graph, (utility), (regex), (format), (thread).}
\item {\bf almost required:} {\bf blas} and {\bf lapack} libraries provide linear algebra
  subroutines that several learning algorithms depend upon. There are standard
  packages (RPMS\ldots) for most Linux distributions, and come as part of
  the {\em veclib} framework under Mac OS X.
\item {\bf recommended:} {\bf ncurses} is used in some places for text mode
  gui.
\item {\bf optional:} {\bf python runtime} is needed for embedding python
  in your plearn executable if you need them to evaluate python code snippets.
\item {\bf optional:} PLearn has an interface to call the {\bf Torch}
  library, which offers a number of additional learning algorithms (in particular SVMs). 
\item {\bf optional:} The libraries of {\bf WordNet} for work on language
  models.
\item {\bf optional:} {\bf MPI} libraries for parallelization.
\end{itemize}

\section{Python related dependencies}

Nearly all higher level GUI demos and plotting facilities are written in
python (which calls upon plearn) using essentially the following software
packages: \\
\begin{itemize}
\item {\bf numarray} for efficient numeric array operations in python.
\item {\bf matplotlib} for 2D plots.
\item {\bf mayavi} for 3D interactive plots.
\item {\bf pygtk} with {\bf gtk+2} for sophisticated GUIs.
\item {\bf tkinter} for older and simpler GUIs.
\end{itemize}

You may also want to install the more user-friedly python command-line interface {\bf
 ipython}: \\
{\tt ipython --pylab} starts a good environment for scientific computing in python.


\subsection{Other useful external tools}

In addition the following tools can be useful:

\begin{itemize}
\item {\bf gs} (ghostscript) and {\bf gv} (ghostview)
\item {\bf gnuplot} (for older plot commands called from plearn)
\item {\bf perl} to run older dirty scripts.
\item {\bf graphviz} to plot graphs.
\end{itemize}

%%%%%%%%%%%%%%%%%%%%%%%%%%%%%%%%%%%%%%%%
\chapter{Downloading PLearn}

After having been hosted for several years on SourceForge (under CVS), as
of 2005/06/21 PLearn development has been moved to BerliOS
(\url{http://developer.berlios.de}) that offered the benefits of a faster
hosting with the more modern Subversion (SVN) version control system.
The latest version is always available through SVN access.  

\section{Anonymous SVN checkout}

If you just want to hack PLearn locally, you can do an anonymous checkout (no need for a BerliOS account) with
\begin{verbatim}
svn checkout svn://svn.berlios.de/plearn/trunk PLearn
\end{verbatim}

\section{Developer SVN access}
If you are going to be a serious contributor to PLearn, you should create a
BerliOS account for yourself, and ask to be added to the developer list.
This will give you read/write access to the PLearn repository.

Make sure you first move any older version of PLearn out of the way, for ex. by renaming it PLearn.old
You can then check-out a fresh copy of PLearn with the following instruction:
\begin{verbatim}
svn checkout svn+ssh://USERNAME@svn.berlios.de/svnroot/repos/plearn/trunk PLearn
\end{verbatim}
where you replace "USERNAME" by your Berlios username.
You will be asked for your Berlios password twice.

If you don't want to be bothered with svn asking passwords, clisk on {\em
  Account Maintenance} on the left panel of your BerliOS personal page, and
on {\em Edit keys} on the bottom of the Maintenance page.  You can copy
your ssh public key there. (Note: your ssh public key is normally found in
your ~/.ssh/*.pub If it's not there, you can do a ssh-keygen). As for many
changes within BerliOS, it may take a while before this is propagated and
taken into account.

We also suggest that you edit your \verb!~/.subversion/config! and look for
the line containing \verb!global-ignores!: uncomment it and add OBJS in the list
of ignored patterns, to avoid being annoyed with these directories when
doing a svn status command. You'll also have to uncomment the line
\verb![miscellany]! three lines above.


% Sub-projects at your institution may still be under CVS.
% For a quick tutorial on how to use cvs, see http://plearn.sourceforge.net/plearncvs.html

\section{SVN basics}

From within your local PLearn directory:
\begin{itemize}
\item {\tt svn update} will update your local version with the latest
  version in the repository.
\item {\tt svn commit} will commit your changes to the repository.
\item {\tt svn add} will add to the repository the files and directories
  you pass as argument, {\bf recursively}, so make sure you {\em really} want to
  add all those directories' {\em full} content\ldots
\item There is also {\tt svn rm, svn mv,} and {\tt svn cp} to reorganize the files.
\end{itemize}

Unlike CVS, most subversion commands are recursive by default. Check the
help for a particular command before using it if you are unsure.

For more details, there is a excellent free subversion book online available at:
http://svnbook.red-bean.com/en/1.1/index.html

If you don't have the time to at least peruse the whole book, I would still
strongly recommend that you at least read appendix A: Subversion for CVS
users: http://svnbook.red-bean.com/en/1.1/apa.html


\section{Overview of the directory structure}

Your checked out PLearn has the following high level directory structure:

\begin{itemize}
\item \verb!scripts/! \\
contains mostly python and perl scripts
  \begin{itemize}
  \item {\tt pymake} is our build system 
  \item {\tt pytest} implements our test suite framework
  \item {\tt pyskeleton} is our ``wizard'' or ``template'' system;  it uses source code templates in the {\tt Skeleton/} directory
  \item {\tt pypoints} allows graphical interactive input of points (for 2D classif or 1D regression)
  \item {\tt pyplot} allows plotting learner outputs (classif decision surface, density plot, etc...) 
  \item {\tt xpdir} lists the content of PLearn experiment directories
  \item {\tt perlgrep}, {\tt search-replace}, and {\tt undo-search-replace} allow for simple code lookup and transformations.
  \item {\tt pytansform} and the transformations in {\tt Transformations/} allow for complex code transformations.
  \item {\tt cvschangeroot} is used to recursively change the locally
  recorded cvs root. This is useful to indicate that the repository has
  moved. Note however that we now use a SVN repository for PLearn rather
  than CVS.
  \item \ldots
  \end{itemize}
\item \verb!commands/! contains the source code (as {\tt .cc} files) of PLearn "executables" to be
compiled with {\tt pymake}. If you look at the \verb!plearn_*.cc! files closely, you will
notice that they are all built in the same manner, simply including a
number of things they need, and invoking a single function
\verb!plearn_main!. So they only differ in the functionalities they
\#include. They have been arranged according to the external
library dependencies that will be needed to compile and link each of them. In short:
\begin{itemize}
\item \verb!plearn_noblas!   depends only on NSPR, boost  (if you don't have blas, it must be compiled with \verb!pymake -noblas plearn_noblas! ) 
\item \verb!plearn_lapack!   depends on NSPR, boost, BLAS, LAPACK
\item \verb!plearn_curses!   depends on NSPR, boost, BLAS, LAPACK, ncurses
\item \verb!plearn_python!   depends on NSPR, boost, BLAS, LAPACK, ncurses, python runtime libraries
\end{itemize}
Since \verb!plearn_noblas! has the smallest number of requirements, it should be the
easiest to get to compile and link. But several important learning algorithms require LAPACK. 
\verb!plearn_lapack! will contain the most useful 99\% of PLearn. 
\item \verb!commands/PLearnCommands/! contains the source code for all PLearn
{\em commands}. These commands are included in the above \verb!plearn_*!
programs and can be invoked with these programs in a command-line fashion.
\item \verb!commands/language/! contains some programs for manipulating language
corpus and WordNet related stuff.
\item \verb!python_modules/! \\
The root for python module namespace. Must be in your \verb!PYTHONPATH!. \\
It contains mostly a \verb!plearn/! subdirectory, which allows to \verb!import plearn.foobar! from python.
\item \verb!doc/! \\
directory within which Pearn's documentation is generated
\item \verb!examples/! \\
contains examples of PLearn scripts 
\item \verb!test_suite! \\
contains part of the PLearn test suite
\item \verb!plearn/! \\
contains all of PLearn's base C++ classes organised in themed subdirectories.
\item \verb!plearn_learners/! \\
contains all of PLearn's C++ learner classes organised in themed subdirectories.
\item \verb!plearn_learners_experimental/! \\
may contain some {\em very experimental} C++ learner classes\ldots
\item \verb!plearn_torch/! \\
contains C++ classes to interface with the Torch library ( http://www.torch.ch/ )
\item \verb!pylearn/! \\
is the beginning of a BoostPython interface to access PLearn from python,
but is currently not the way we call upon PLearn from python. 
\end{itemize}

The \verb!plearn*! directories contain C++ source code ({\tt .h} and {\tt
  .cc}), and as your root PLearn directory is in the \verb!-I! directive of
the compilation commands, this allows to include relevant files by
directives such as, for ex:
\begin{verbatim}
#include<plearn/base/Object.h>
#include<plearn/io/PStream.h>
#include<plearn_learners/generic/PLearner.h>
\end{verbatim}



\chapter{Installing PLearn}

\section{Installation on Linux}

\subsection{PLearn setup and compilation}

We suppose you have all the necessary software requirements in place (especially python).

For ex. under ubuntu, you can check the following packages install (using
synaptic for ex.):

Required:
{\tt g++, python2.4, python, libnspr4, libnspr-dev, libboost-dev,
libboost-regex1.33.1, libboost-regex-dev, libboost-graph1.33.1, libboost-graph-dev} \\
Strongly suggested: \verb!libncurses5, libncurses5-dev! and some version of lapack and blas. Ex:
{\tt refblas3, lapack3, lapack3-dev} \\
Recommended (python and graphics related):
{\tt ipython, python-matplotlib, 
python-numarray, python-numpy, mayavi, python2.4-dev,
python-tk, python-gtk2, python-gtkhtml2} \\

Then edit your .cshrc or .bashrc and 
\begin{itemize}
\item Set the {\tt PLEARNDIR} environment variable to the path of your PLearn directory. 
(csh ex: \verb!setenv PLEARNDIR ${HOME}/PLearn!)
\item Append \verb!$PLEARNDIR/scripts! and \verb!$PLEARNDIR/commands! to your path.
\item Append \verb!$PLEARNDIR/python_modules! to your \verb!PYTHONPATH!
\item Restart your shell for these changes to take effect.
\end{itemize}

From within your PLearn directory run {\tt ./setup}. This should create a
{\tt .plearn} sub-directory in your home directory, that will contain some
configuration files.

Now you should be able to try and compile a first version of plearn.
We have our own make system based on a python script ({\tt pymake}) that
automatically parses source files for dependencies and determines what to
compile, and what to link (including optional libraries), and is able to
run parallel compilation on several machines.  It is easily
customizable.

The compiling commands are:
\begin{verbatim}
cd PLearn/commands/
pymake -noblas plearn_noblas.cc
\end{verbatim}

If it doesn't work, you may have to adapt the configuration file to your system
(PLearn/.pymake/config)

If it does work, you can try to \verb!pymake plearn_lapack.cc! or even
\verb!pymake plearn_curses.cc!


\section{Installation on Mac OS X}

\subsection{External dependencies}
If you want to see graphical displays, you should also install X11 (apple
has a version on its system install CDS shipped with the computers as part
of XCode: look for packages {\bf X11 User} and {\bf X11 SDK}.

You should then install {\bf fink} ( \url{http://fink.sourceforge.net} )
and its GUI {\bf Fink Commander} ( \url{http://finkcommander.sourceforge.net/} ).
We recommend you also enable the {\em use of unstable package} (in {\tt Menu Fink
Commander / Preferences.../ Fink} ) to gain access to the latest packages.

To be able to compile and link the core of plearn, you should install the
following packages through fink:
\begin{itemize}
\item {\tt python24}
\item {\tt boost1.32-py24}
\item {\tt mozilla-dev} (for NSPR)
\end{itemize}

Optional libraries also easily installable from fink:
\begin{itemize}
\item {\tt ncurses}  (useful for viewing dataset tables)
\item {\tt numarray-py24} for efficient matrix/vector manipulations in python
\item {\tt matplotlib-py24} for 2D graphics
\item {\tt mayavi-py24} for 3D interactive graphics
\item {\tt pygtk2-py24} if you want to play with the
  python GUI tools and demo.
\end{itemize}

% NOTE: I couldn't get plide to run on Mac OS X, due to missing gtkhtml2
% dependency. It seems that this dependency might be in a
% gnome-python2-extras-py24 package, which is not yet available on fink, and
% which I couldn't get to compile and link from the source
% http://ftp.gnome.org/pub/GNOME/sources/gnome-python-extras/

\subsection{Environment setup}

You should make sure the following variables and paths are correctly defined in your
environment variables (in your {\tt .profile} with your {\tt .bashrc}
simply doing a \verb!source $HOME/.profile!)

\begin{verbatim}
# The default fink installs its packages in /sw and should already have
# added the following line.
source /sw/bin/init.sh

# adapt to your configuration if your PLearn directory is not $HOME/PLearn
export PLEARNDIR=$HOME/PLearn
export PATH=/sw/bin:/sw/sbin:$PLEARNDIR/scripts:$PLEARNDIR/commands:.:$PATH
export LD_LIBRARY_PATH="/sw/lib:/sw/lib/mozilla:$LD_LIBRARY_PATH"
export LIBRARY_PATH=$LD_LIBRARY_PATH
export CPATH="/sw/include:/sw/include/mozilla:$CPATH"
export PYTHONPATH="$PLEARNDIR/python_modules:$PLEARNDIR/scripts:$PYTHONPATH"
export SKELETONS_PATH=$PLEARNDIR/scripts/Skeletons
\end{verbatim}

Restart a new shell for these to take effect. Make sure they're well
defined in the new shell.

\subsection{PLearn setup and compilation}

From within your PLearn directory run {\tt ./setup}. This should create a
{\tt .plearn} sub-directory in your home directory, that will contain some
configuration files.

The compiling commands are:
\begin{verbatim}
cd PLearn/commands/
pymake -noblas plearn_noblas.cc
\end{verbatim}

If it doesn't work, you may have to adapt the configuration file to your system
(PLearn/.pymake/config)

If it does work, you can try to \verb!pymake plearn_lapack.cc! or even
\verb!pymake plearn_curses.cc!

\section{Installation on Windows with cygwin}
\label{sec:windows}

This section describes a step-by-step installation under the Microsoft Windows environment.
Note that the following instructions are outdated.

Cygwin (\url{http://cygwin.com}) is a Linux-like environment for Windows, and is
currently the easiest route to using PLearn under Windows.

\subsection{Installing Cygwin}
Download from \url{http://cygwin.com/setup.exe} the latest Cygwin setup program,
then run it.
Select your installation options (you should keep the recommended Unix / binary default
text file type).
Once you reach the ``Select Packages'' step, click the ``View'' button to switch
to full view and select the following packages to install (the version number in
parenthesis was the version used when this guide was written, hopefully any further
version should work too).

\begin{itemize}
\item \verb!autoconf! (2.59-2)
\item \verb!gcc-g++! (3.4.4-1)
\item \verb!gcc-g77! (3.4.4-1)
\item \verb!lapack! (3.0-3)
\item \verb!libncurses-devel! (5.4-4)
\item \verb!make! (3.80-1)
\item \verb!python! (2.4.1-1)
\item \verb!subversion! (1.2.3-1)
\end{itemize}

Optional (but recommended) packages to install:
\begin{itemize}
\item \verb!gdb! (20041228-3)
\item \verb!tcsh! (6.14.00-5)
\item \verb!unzip! (5.50-5)
\item \verb!vim! (6.4-2)
\end{itemize}

The default shell installed with Cygwin is bash, but in the following, we will
be using tcsh (though you may of course adapt the instructions below to get PLearn to
work with bash). Assuming you have installed the tcsh package and Cygwin is
installed in \verb!C:\cygwin!, edit \verb!C:\cygwin\cygwin.bat! and replace
the line \verb!bash --login -i! with \verb!tcsh -l!.
You may also change the default location of your home directory by adding the
following line at the beginning of the file (make sure you do not have blanks
in the path you provide):
\begin{verbatim}
@SET HOME=C:\MyCygwinHome
\end{verbatim}

Note that it is suggested to use a ``low-level'' path for your home directory, i.e. as close
to the root as possible, because of the limitations of Windows concerning the
length of paths (which may cause some tests to fail in the test-suite).

\subsection{Installing Boost}
The next step is to install the Boost library.
You could install part of it from the Cygwin setup utility, but you would
neeed to compile Boost-Python anyway.
Go to \url{http://sourceforge.net/projects/boost/} then to the ``Files'' section
to download
the Boost library. Download the \verb!tar.gz! files:
you will need the source for both Boost and Boost-Jam (the Boost installer).
Move these files to your Cygwin home directory.
The files used when writing this guide were \verb!boost_1_33_1.tar.gz!
and \verb!boost-jam-3.1.11.tgz!.

You may now run Cygwin and launch the Boost installation:
\begin{enumerate}
\item \verb!tar zxvf boost_1_33_1.tar.gz!
\item \verb!tar zxvf boost-jam-3.1.11.tgz!
\item \verb!cd boost-jam-3.1.11!
\item \verb!sh ./build.sh!
\item \verb!cd ../boost_1_33_1!
\item \verb!../boost-jam-3.1.11/bin.cygwinx86/bjam.exe -sTOOLS=gcc! \\
\verb!--prefix=$HOME/local -sPYTHON_ROOT=/usr! \\
\verb! -sPYTHON_VERSION=2.4 --builddir=/tmp/boost_build install!
\end{enumerate}

The installation command above will install Boost in the \verb!local! directory
under your home directory: you may remove the \verb!--prefix! option to perform
a system-wide installation (however, installing under your home directory is often
necessary for Linux users without administrator rights).
Do not worry about some of the Boost libraries not compiling, PLearn does not
need all of them.
Let us just create the appropriate links for those needed by PLearn:
\begin{enumerate}
\item \verb!cd ~/local/lib!
\item \verb!ln -s libboost_regex-gcc-mt-1_33_1.dll libboost_regex.dll!
\item \verb!ln -s libboost_python-gcc-mt-1_33_1.dll libboost_python.dll!
\item \verb!cd ~/local/include!
\item \verb!ln -s boost-1_33_1/boost!
\end{enumerate}

\subsubsection{Installing NumArray}

Download NumArray from \\
\url{http://www.stsci.edu/resources/software\_hardware/numarray} \\
then install it with the following commands:
\begin{verbatim}
tar zxvf numarray-1.5.0.tar.gz
cd numarray-1.5.0
python setup.py config install --gencode --prefix=$HOME/local
\end{verbatim}

Note that (at least in version 1.5.0) there is a typo in a NumArray C++ file.
After you have installed NumArray with the commands above, edit
\verb!~/local/include/python2.4/numarray/arraybase.h! to manually remove the comma
at the end of line 117. Additionally, if you want to get rid of a gcc warning
when compiling PLearn, you can edit \verb!libnumarray.h!
in the same directory and replace line 51 with
\begin{verbatim}
static void **libnumarray_API __attribute__ ((unused)) ;
\end{verbatim}

\subsection{Installing NSPR}

Go to \verb!ftp://ftp.mozilla.org/pub/mozilla.org/nspr/releases! to get
the NSPR release for your operating system (in this guide we used the
4.6 release for Windows NT 5.0, in optimized mode).
Extract the zip file to a temporary directory, then move the content of the
\verb!include! directory (including its sub-directories) to \verb!~/local/include/mozilla/nspr! (that you
will need to create), and the
\verb!libnspr4.dll! file (in the \verb!lib! directory) to \verb!~/local/lib!.
Check the permissions for \verb!libnspr4.dll!: you need the ``execute'' permission
for PLearn to be able to run. You can set it with:\\
\verb!chmod u+x ~/local/lib/libnspr4.dll!


\subsection{Environment setup}

First, go to the directory
where you wish to install PLearn (in this guide we will assume this is your
home directory), and check out the latest
version from the Subversion repository:\\
\verb!svn checkout svn://svn.berlios.de/plearn/trunk PLearn!

Since \verb!pymake! will create directories named \verb!OBJS! to store
compiled object files, Subversion should ignore them: edit
your \verb!~/.subversion/config! and, in the \verb!miscellany! section,
write the line\\
\verb!global-ignores = OBJS!

Now, edit your \verb!~/.cshrc! and add the following lines:

\begin{verbatim}
# Environment variables.
setenv PLEARNDIR ${HOME}/PLearn
setenv PATH /usr/local/bin:/usr/bin:/bin:/usr/lib/lapack:
            ${PLEARNDIR}/commands:${PLEARNDIR}/scripts:
            ${HOME}/local/lib
setenv PYTHONPATH ${PLEARNDIR}/python_modules:
                  ${HOME}/local/lib/python2.4/site-packages
setenv CPATH ${HOME}/local/include
setenv LD_LIBRARY_PATH ${HOME}/local/lib
setenv LIBRARY_PATH ${LD_LIBRARY_PATH}

# Nicer prompt.
set prompt = "%B%m %~%b > "
\end{verbatim}

You need to redefine the \verb!PATH! environment variable because the
original one will usually contain directories with blanks (such as
\verb!Program Files!), which Cygwin has trouble with.
The last line is very optional (it just gives you a nicer prompt).
Now edit \verb!$PLEARNDIR/pymake.config.model! and look for the
\verb!python! optional library.
Just before, add the following lines:
\begin{verbatim}
python_version = '2.4'
compileflags += ' -DPL_PYTHON_VERSION=240'
python_lib_root = '/usr/lib'
numpy_site_packages = join(homedir,
              'local/lib/python2.4/site-packages/numarray')
\end{verbatim}

Do a \verb!source ~/.cshrc! to reload your configuration file, then go
to your PLearn installation directory (\verb!$PLEARNDIR!) and run \verb!./setup!. PLearn can now be compiled
with \verb!pymake $PLEARNDIR/commands/plearn!.



\chapter{Other installation-related information}

\section{Testing PLearn}

Once PLearn compiles successfully, it is recommended to run the test-suite
in order to ensure that the main components of the library work as expected.
Go to \verb!$PLEARNDIR! and launch \verb!pytest run --all!.
Note that, typically, the test-suite first compiles \verb!$PLEARNDIR/commands/plearn_tests!,
and thus may look like it is stalled, while it is actually compiling or linking
in the background.

\section{Generating the documentation locally}

The {\tt doc} subdirectory maintains the sources of the documentation that is also
available on the PLearn website. To generate it locally from source you'll
need the following software tools:
\begin{itemize}
\item A working LaTeX distribution such as {\tt teTeX}
\item {\tt latex2html}
\item {\tt doxygen}
\end{itemize}



\chapter*{License}

This document is covered by the license appearing after the title page.

\vspace*{.5cm}

The PLearn software library and tools described in this document are
distributed under the following BSD-type license:

\begin{verbatim}
Redistribution and use in source and binary forms, with or without
modification, are permitted provided that the following conditions are met:
 
  1. Redistributions of source code must retain the above copyright
     notice, this list of conditions and the following disclaimer.
 
  2. Redistributions in binary form must reproduce the above copyright
     notice, this list of conditions and the following disclaimer in the
     documentation and/or other materials provided with the distribution.
 
  3. The name of the authors may not be used to endorse or promote
     products derived from this software without specific prior written
     permission.
 
 THIS SOFTWARE IS PROVIDED BY THE AUTHORS ``AS IS'' AND ANY EXPRESS OR
 IMPLIED WARRANTIES, INCLUDING, BUT NOT LIMITED TO, THE IMPLIED WARRANTIES
 OF MERCHANTABILITY AND FITNESS FOR A PARTICULAR PURPOSE ARE DISCLAIMED. IN
 NO EVENT SHALL THE AUTHORS BE LIABLE FOR ANY DIRECT, INDIRECT, INCIDENTAL,
 SPECIAL, EXEMPLARY, OR CONSEQUENTIAL DAMAGES (INCLUDING, BUT NOT LIMITED
 TO, PROCUREMENT OF SUBSTITUTE GOODS OR SERVICES; LOSS OF USE, DATA, OR
 PROFITS; OR BUSINESS INTERRUPTION) HOWEVER CAUSED AND ON ANY THEORY OF
 LIABILITY, WHETHER IN CONTRACT, STRICT LIABILITY, OR TORT (INCLUDING
 NEGLIGENCE OR OTHERWISE) ARISING IN ANY WAY OUT OF THE USE OF THIS
 SOFTWARE, EVEN IF ADVISED OF THE POSSIBILITY OF SUCH DAMAGE.
\end{verbatim}

\end{document}
