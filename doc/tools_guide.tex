%% -*- mode:latex; tex-open-quote:"\\og{}"; tex-close-quote:"\\fg{}" -*-
%%
%%  Copyright (c) 2002 by Pascal Vincent
%%
%%  $Id$

\documentclass[11pt]{book}
\usepackage{html}               % package for latex2html
\usepackage{hyperref}
%\usepackage{t1enc}              % new font encoding  (hyphenate words w/accents)
%\usepackage{ae}                 % use virtual fonts for getting good PDF
% \usepackage{isolatin1}		% support for French accents

%%%%%%%%% Definitions %%%%%%%%%%%%
\newcommand{\PLearn}{\bf \it PLearn}
\newcommand{\Object}{\bf Object} 
\newcommand{\Learner}{\bf Learner} 
\newcommand{\PPointable}{\bf PPointable} 

\parskip=2mm
\parindent=0mm

\begin{document}

%%%%%%%% Title Page %%%%%%%%%%
\pagenumbering{roman}
\thispagestyle{empty}

\thispagestyle{empty}
\begin{center}
{\Huge PLearn Programmer's Tools Guide}\\
\vspace{.5cm}
{\Large The tools that make the programmer's life simpler}\\ 
\vspace{6.5in}
\end{center}
\pagebreak

\vspace*{10cm}


{\small

Copyright \copyright\ 2002 Pascal Vincent \\

Permission is granted to copy and distribute this document in any medium,
with or without modification, provided that the following conditions are
met:

\begin{enumerate}
\item Modified versions must give fair credit to all authors.
\item Modified versions may not be written with the aim to discredit, misrepresent, or otherwise taint the
      reputation of any of the above authors.
\item Modified versions must retain the above copyright notice, and append to
   it the names of the authors of the modifications, together with the years the
   modifications were written.
\item Modified versions must retain this list of conditions unaltered, 
    and may not impose any further restrictions.
\end{enumerate}
}

\pagebreak

%%%%%%%%% Table of contents %%%%%%%%%%%%
\addcontentsline{toc}{chapter}{\numberline{}Table of contents}
\tableofcontents

\cleardoublepage\pagebreak
\pagenumbering{arabic}

%%%%%%%%%%%%%%%%%%%%%%%%%%%%%%%%%%%%%%%%%%%%%%%%%%%%%%%%%%%%%%%%%%%%%%%%%%%%%%%

\chapter{PLearn scripts}

%% \section{upackage}

%% {\bf upackage} is a python script (together with a set of python modules)
%% that can be used to easily install libraries and other tools on your
%% system.  It was created in order to simplify the installation of external
%% components on which PLearn depends, but it is a general packaging system
%% (usage is similar to {\em apt-get} or {\em emerge}) that may be used to install
%% about anything.

%% The {\bf u} in {\bf upackage} stands for {\bf u}ser, as you don'y need to
%%   be root to install stuff with {\bf upackage}.

%% Other typical packaging systems work under a number of assumptions:
%% \begin{itemize}
%% \item The person installing the package must have administrative rights
%% \item Dependency checking is done by requiring other packages \emph{of
%%   the same packaging system} to be installed (usually relying on a
%%   local database listing previously installed packages).
%% \item They are restricted to a single operating-system, even worse: to a specific
%%   distribution of that operating system. 
%% \end{itemize}

%% If you are a user then there are two possibilities. Either you are in the
%% ideal situation where somebody has written a package for your {\em
%% specific} os and distribution, and you are a very good friend of an {\em
%% efficient and dedicated} system administrator (not the overworked kind that
%% will "look into the matter when I have time"). Or you are not in
%% that ideal situation. Then you can always install by yourself from
%% source\ldots And you know how the music goes: search the web for some
%% install doc, find where to download the archive from, unpack, try to
%% compile, understand what is going wrong, patch some files, find out missing
%% dependencies and start over again for each of those: recurse!

%% We don't want people willing to try out plearn to have such a huge barrier to
%% entry. Hence {\bf upackage} is essentially an attempt to automate what
%% you'd do if installing all the dependencies by hand.

%% In other words: {\bf upackage} aims at becoming an automated non-root
%% cross-distribution and even cross-platform package installation system.


\section{pymake}

\section{pyskeleton}

\section{perlgrep}

\section{search-replace}

\section{pytransform}

\section{cvschangeroot}

%%%%%%%%%%%%%%%%%%%%%%%%%%%%%%%%%%%%%%%%%%%%%%%%%%%%%%%%%%%%%%%%%%%%%%%%%%

\chapter{The PLearn test-suite }

%{\em This article is a stub. You can help PLearn by expanding it.}
%(\copyright Wikipedia)

PLearn test-suite is still in development by its author Christian
Dorion, but it is already usable and should probably be used by anyone
who wants to ensure his code does not get broken accidentally.

Each {\em test} is a program or a PLearn script, which process some
files (such as datasets), and outputs something to the standard output
and error, and possibly to other files. The goal of the test-suite is to
compare these results, run with a recently-compiled copy of PLearn, to
the reference results created by the first run of the test.

The following instructions are a step-by-step example, for a test named
{\tt PL\_Var\_utils}, testing the {\tt Var} class. It is not officially
supported by Christian, and may not work anymore on a later version of
the test-suite.

In addition, those who already know of the test-suite may find the
PLearn object class {\tt PTest} useful to write C++ tests that do not
make the test-suite explode both in size and execution time (in addition
to provide understandable floating numbers diff through the {\tt PLearn
diff} command applied on objects).


\begin{enumerate}
  \item If it does not already exist, create the appropriate {\tt test}
  directory, add this directory to the Subversion repository, and move
  into it:
\begin{verbatim}
$ mkdir ${PLEARNDIR}/plearn/var/test
$ svn add ${PLEARNDIR}/plearn/var/test
$ cd ${PLEARNDIR}/plearn/var/test
\end{verbatim}

  \item {\em [Optional]} If your test relies on more than a couple
  of custom files (data files, scripts, ...), better create its own
  directory:
\begin{verbatim}
$ mkdir Var_utils
$ svn add Var_utils
$ cd Var_utils
\end{verbatim}

  \item If it does not already exist, create a template {\tt
  pytest.config} file:
\begin{verbatim}
$ pytest add
\end{verbatim}

  \item There are mainly two kinds of tests. If a simple {\tt .plearn}
  or {\tt .pyplearn} script is enough to run the test (e.g., if you plan
  to test a new PLearner class), go to step \ref{small_test}. If you need to create
  C++ code to test some specific functions, go on to step \ref{big_test}.

  \item \label{big_test} Create a PTest subclass template:
\begin{verbatim}
$ pyskeleton PTest VarUtilsTest
\end{verbatim}

  \item Edit the resulting files (e.g., {\tt VarUtilsTest.h} and {\tt
VarUtilsTest.cc}):
  \begin{itemize}
    \item fill the {\tt PLEARN\_IMPLEMENT\_OBJECT} macro help:
\begin{verbatim}
PLEARN_IMPLEMENT_OBJECT(
    arUtilsTest,
    "Test various functions in Var_utils",
    ""
);
\end{verbatim}

    \item write your actual test code in the {\tt perform()} method
    \item store the test results, you can either:
    \begin{enumerate}
      \item display them (using the {\tt pout} or {\tt perr} PStreams,
      or the PLearn logging system):
\begin{verbatim}
// In VarUtilsTest::perform
pout << my_function(x) << endl;
MAND_LOG << my_function(x) << endl;
\end{verbatim}

      \item store them in your {\tt PTest} object options (this requires
      a little more work, but is actually easier to understand when the
      test fails):
\begin{verbatim}
// In VarUtilsTest.h
map<string, Vec> vec_results;
// In VarUtilsTest::declareOptions
declareOption(ol, "vec_results",
              &VarUtilsTest::vec_results,
              OptionBase::learntoption,
              "Test Vec results.");
// In VarUtilsTest::perform
vec_results["my_function"] = my_function(x);
\end{verbatim}

      \item remember that the text (and PLearn binary-formatted)
      files that might be output by the program are also compared (no
      need to output them to {\tt cout}).
    \end{enumerate}
  \end{itemize}

  \item Once your code compiles and is ready to be tested, add your new
  PTest in {\tt PLearn/commands/plearn\_tests\_inc.h}:
\begin{verbatim}
#include <plearn/var/test/VarUtilsTest.h>
\end{verbatim}

  \item
  Edit your {\tt pytest.config} file to specify how your test is
  supposed to be run. The name of your test should begin by ``PL\_'' if
  it is a normal PLearn test.

    Typically you will run {\tt plearn\_tests}, on a {\tt .plearn}
    or {\tt .pyplearn} script (this script will describe one or more
    objects of your PTest subclass, with its specific options):
\begin{verbatim}
Test(
    name = "PL_Var_util",
    description = "Test various functions in Var_utils",
    program = GlobalCompilableProgram(
        name = "plearn_tests",
        compiler = "pymake",
        compile_options = ""
    ),
    arguments = "varutils_test.plearn",
    resources = [ "varutils_test.plearn" ],
    precision = 1e-06,
    disabled = False
)
\end{verbatim}

    But if your {\tt PTest} object does not need extra options,
    you can save the use of a script:
\begin{verbatim}
Test(
    name = "PL_Var_util",
    description = "Test various functions in Var_utils",
    program = GlobalCompilableProgram(
        name = "plearn",
        compiler = "pymake",
        compile_options = ""
    ),
    arguments = "PLEARNDIR:scripts/command_line_object.plearn " \
        "'object=VarUtilsTest()'",
    resources = [ ],
    precision = 1e-06,
    disabled = False
)
\end{verbatim}

  \item \label{debug} {\em [Optional]} For debug purpose, you may temporarily use a
  copy (say {\tt plearn\_mytests}) of {\tt plearn\_tests} in {\tt
  pytest.config}, and also comment out all tests but your own test in
  {\tt plearn\_mytests\_inc.h}. This will make compilation faster when
  debugging your test. Setting {\tt compile\_options} to {\tt -opt} will
  also probably speed up the link time.

  \item Run your test to generate results. Check everything is fine in
  the generated {\tt .pytest/expected\_results} hidden directory. If it
  is not, fix your code. Do this until you are happy with the results.
\begin{verbatim}
$ pytest results -n PL_Var_util
\end{verbatim}

  \item If you had made any of the actions described in optional step
  \ref{debug}, revert back to the standard test configuration.

  \item Add your specific files to version control:
\begin{verbatim}
$ svn add VarUtilsTest.h VarUtilsTest.cc
\end{verbatim}

  \item Go to step \ref{run}.

  \item \label{small_test} Write your {\tt .plearn} or {\tt .pyplearn}
  script, and make sure it runs smoothly.

  \item Once you are confident it works fine, remove all the data that
  was generated while running it (the script must generate data files,
  or output to {\tt cout} or {\tt cerr}, otherwise it is useless).

  \item Edit your {pytest.config} file to specify how your test is to be
  run:
\begin{verbatim}
Test(
    name = "PL_Var_util",
    description = "Test various functions in Var_utils",
    program = GlobalCompilableProgram(
        name = "plearn",
        compiler = "pymake",
        compile_options = ""
    ),
    arguments = "varutils_test.plearn",
    resources = [ "varutils_test.plearn" ],
    precision = 1e-06,
    disabled = False
)
\end{verbatim}

  \item Run your test to generate results. Check everything is fine in
  the generated {\tt .pytest/expected\_results} hidden directory. If it
  is not, fix your script. Do this until you are happy with the results.
\begin{verbatim}
$ pytest results -n PL_Var_util
\end{verbatim}

  \item Add your specific files to version control:
\begin{verbatim}
$ svn add varutils_test.plearn
\end{verbatim}

  \item \label{run} Check that your test works fine:
\begin{verbatim}
$ pytest run
\end{verbatim}

  \item \label{confirm} Confirm that the results obtained are correct
  for the current test. This will {\tt svn add} the result files, and
  put the {\tt svn:ignore} property on some files, but nothing will be
  committed yet.
\begin{verbatim}
$ pytest confirm
\end{verbatim}

  \item Do a {\tt svn status} to check what files have been generated
  but should be ignored by version control, and ignore them:
\begin{verbatim}
$ svn propedit svn:ignore .pytest
\end{verbatim}
  And in the editor it opens, type:
\begin{verbatim}
*.compilation_log
run_results
\end{verbatim}

  \item Assuming your test passes correctly, it is ready for commit:
\begin{verbatim}
$ cd ${PLEARNDIR}
$ svn status          <-- to check which files you need to commit
$ svn commit commands/plearn_tests_inc.h plearn/var/test \
    -m "New test: VarUtilsTest"
\end{verbatim}
\end{enumerate}

%%%%%%%%%%%%%%%%%%%%%%%%%%%%%%%%%%%%%%%%%%%%%%%%%%%%%%%%%%%%%%%%%%%%%%%%%

\chapter{The speed benchmark suite}

%%%%%%%%%%%%%%%%%%%%%%%%%%%%%%%%%%%%%%%%%%%%%%%%%%%%%%%%%%%%%%%%%%%%%%%%%

\chapter{External tools}

\section{cvs}

\section{valgrind}

\section{doxygen}

\section{latex2html}

\section{mpi}


\chapter*{License}

This document is covered by the license appearing after the title page.

\vspace*{.5cm}

The PLearn software library and tools described in this document are
distributed under the following BSD-type license:

\begin{verbatim}
Redistribution and use in source and binary forms, with or without
modification, are permitted provided that the following conditions are met:
 
  1. Redistributions of source code must retain the above copyright
     notice, this list of conditions and the following disclaimer.
 
  2. Redistributions in binary form must reproduce the above copyright
     notice, this list of conditions and the following disclaimer in the
     documentation and/or other materials provided with the distribution.
 
  3. The name of the authors may not be used to endorse or promote
     products derived from this software without specific prior written
     permission.
 
 THIS SOFTWARE IS PROVIDED BY THE AUTHORS ``AS IS'' AND ANY EXPRESS OR
 IMPLIED WARRANTIES, INCLUDING, BUT NOT LIMITED TO, THE IMPLIED WARRANTIES
 OF MERCHANTABILITY AND FITNESS FOR A PARTICULAR PURPOSE ARE DISCLAIMED. IN
 NO EVENT SHALL THE AUTHORS BE LIABLE FOR ANY DIRECT, INDIRECT, INCIDENTAL,
 SPECIAL, EXEMPLARY, OR CONSEQUENTIAL DAMAGES (INCLUDING, BUT NOT LIMITED
 TO, PROCUREMENT OF SUBSTITUTE GOODS OR SERVICES; LOSS OF USE, DATA, OR
 PROFITS; OR BUSINESS INTERRUPTION) HOWEVER CAUSED AND ON ANY THEORY OF
 LIABILITY, WHETHER IN CONTRACT, STRICT LIABILITY, OR TORT (INCLUDING
 NEGLIGENCE OR OTHERWISE) ARISING IN ANY WAY OUT OF THE USE OF THIS
 SOFTWARE, EVEN IF ADVISED OF THE POSSIBILITY OF SUCH DAMAGE.

\end{verbatim}


\end{document}

