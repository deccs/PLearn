%% -*- mode:latex; tex-open-quote:"\\og{}"; tex-close-quote:"\\fg{}" -*-
%%
%%  Copyright (c) 2002 by Pascal Vincent
%%
%%  $Id: tools_guide.tex,v 1.3 2004/08/05 19:29:03 plearner Exp $

\documentclass[11pt]{book}
\usepackage{t1enc}              % new font encoding  (hyphenate words w/accents)
\usepackage{ae}                 % use virtual fonts for getting good PDF
\usepackage{isolatin1}		% support for French accents

%%%%%%%%% Definitions %%%%%%%%%%%%
\newcommand{\PLearn}{\bf \it PLearn}
\newcommand{\Object}{\bf Object} 
\newcommand{\Learner}{\bf Learner} 
\newcommand{\PPointable}{\bf PPointable} 

\parskip=2mm
\parindent=0mm

\begin{document}

%%%%%%%% Title Page %%%%%%%%%%
\pagenumbering{roman}
\thispagestyle{empty}

\thispagestyle{empty}
\begin{center}
{\Huge PLearn Programmer's Tools Guide}\\
\vspace{.5cm}
{\Large The tools that make the programmer's life simpler}\\ 
\vspace{6.5in}
\end{center}
\pagebreak

\vspace*{10cm}


{\small

Copyright \copyright\ 2002 Pascal Vincent \\

Permission is granted to copy and distribute this document in any medium,
with or without modification, provided that the following conditions are
met:

\begin{enumerate}
\item Modified versions must give fair credit to all authors.
\item Modified versions may not be written with the aim to discredit, misrepresent, or otherwise taint the
      reputation of any of the above authors.
\item Modified versions must retain the above copyright notice, and append to
   it the names of the authors of the modifications, together with the years the
   modifications were written.
\item Modified versions must retain this list of conditions unaltered, 
    and may not impose any further restrictions.
\end{enumerate}
}

\pagebreak

%%%%%%%%% Table of contents %%%%%%%%%%%%
\addcontentsline{toc}{chapter}{\numberline{}Table of contents}
\tableofcontents

\cleardoublepage\pagebreak
\pagenumbering{arabic}

%%%%%%%%%%%%%%%%%%%%%%%%%%%%%%%%%%%%%%%%%%%%%%%%%%%%%%%%%%%%%%%%%%%%%%%%%%%%%%%

\chapter{PLearn scripts}

\section{upackage}

{\bf upackage} is a python script (together with a set of python modules)
that can be used to easily install libraries and other tools on your
system.  It was created in order to simplify the installation of external
components on which PLearn depends, but it is a general packaging system
(usage is similar to {\em apt-get} or {\em emerge}) that may be used to install
about anything.

The {\bf u} in {\bf upackage} stands for {\bf u}ser, as you don'y need to
  be root to install stuff with {\bf upackage}.

Other typical packaging systems work under a number of assumptions:
\begin{itemize}
\item The person installing the package must have administrative rights
\item Dependency checking is done by requiring other packages \emph{of
  the same packaging system} to be installed (usually relying on a
  local database listing previously installed packages).
\item They are restricted to a single operating-system, even worse: to a specific
  distribution of that operating system. 
\end{itemize}

If you are a user then there are two possibilities. Either you are in the
ideal situation where somebody has written a package for your {\em
specific} os and distribution, and you are a very good friend of an {\em
efficient and dedicated} system administrator (not the overworked kind that
will "look into the matter when I have time"). Or you are not in
that ideal situation. Then you can always install by yourself from
source\ldots And you know how the music goes: search the web for some
install doc, find where to download the archive from, unpack, try to
compile, understand what is going wrong, patch some files, find out missing
dependencies and start over again for each of those: recurse!

We don't want people willing to try out plearn to have such a huge barrier to
entry. Hence {\bf upackage} is essentially an attempt to automate what
you'd do if installing all the dependencies by hand.

In other words: {\bf upackage} aims at becoming an automated non-root
cross-distribution and even cross-platform package installation system.


\section{pymake}

\section{cvschangeroot}

\section{perlgrep}

\section{search-replace}

\section{pytransform}

%%%%%%%%%%%%%%%%%%%%%%%%%%%%%%%%%%%%%%%%%%%%%%%%%%%%%%%%%%%%%%%%%%%%%%%%%%

\chapter{The PLearn test-suite }

%%%%%%%%%%%%%%%%%%%%%%%%%%%%%%%%%%%%%%%%%%%%%%%%%%%%%%%%%%%%%%%%%%%%%%%%%

\chapter{The speed benchmark suite}

%%%%%%%%%%%%%%%%%%%%%%%%%%%%%%%%%%%%%%%%%%%%%%%%%%%%%%%%%%%%%%%%%%%%%%%%%

\chapter{External tools}

\section{cvs}

\section{valgrind}

\section{doxygen}

\section{latex2html}

\section{mpi}


\chapter*{License}

This document is covered by the license appearing after the title page.

\vspace*{.5cm}

The PLearn software library and tools described in this document are
distributed under the following BSD-type license:

\begin{verbatim}
Redistribution and use in source and binary forms, with or without
modification, are permitted provided that the following conditions are met:
 
  1. Redistributions of source code must retain the above copyright
     notice, this list of conditions and the following disclaimer.
 
  2. Redistributions in binary form must reproduce the above copyright
     notice, this list of conditions and the following disclaimer in the
     documentation and/or other materials provided with the distribution.
 
  3. The name of the authors may not be used to endorse or promote
     products derived from this software without specific prior written
     permission.
 
 THIS SOFTWARE IS PROVIDED BY THE AUTHORS ``AS IS'' AND ANY EXPRESS OR
 IMPLIED WARRANTIES, INCLUDING, BUT NOT LIMITED TO, THE IMPLIED WARRANTIES
 OF MERCHANTABILITY AND FITNESS FOR A PARTICULAR PURPOSE ARE DISCLAIMED. IN
 NO EVENT SHALL THE AUTHORS BE LIABLE FOR ANY DIRECT, INDIRECT, INCIDENTAL,
 SPECIAL, EXEMPLARY, OR CONSEQUENTIAL DAMAGES (INCLUDING, BUT NOT LIMITED
 TO, PROCUREMENT OF SUBSTITUTE GOODS OR SERVICES; LOSS OF USE, DATA, OR
 PROFITS; OR BUSINESS INTERRUPTION) HOWEVER CAUSED AND ON ANY THEORY OF
 LIABILITY, WHETHER IN CONTRACT, STRICT LIABILITY, OR TORT (INCLUDING
 NEGLIGENCE OR OTHERWISE) ARISING IN ANY WAY OUT OF THE USE OF THIS
 SOFTWARE, EVEN IF ADVISED OF THE POSSIBILITY OF SUCH DAMAGE.

\end{verbatim}


\end{document}

